% !TeX TS-program = pdflatexmk
% !BIB TS-program = biber

\documentclass[10pt,letterpaper,article]{memoir} % for a short document

% Bibliography
\usepackage[authordate,strict,babel=other,bibencoding=inputenc,doi=false,isbn=false,annotation=true]{biblatex-chicago}
\addbibresource{bike-counts-refs.bib}

\setlrmarginsandblock{1in}{1in}{*}
\setulmarginsandblock{.5in}{1in}{*}
\checkandfixthelayout

\pagestyle{empty}

\title{CSSS 504 Project Proposal}
\author{Giles Goetz, Robin Gold, Peter Schmiedeskamp}
\date{14 February 2014} % Delete this line to display the current date

%%% BEGIN DOCUMENT
\begin{document}
\maketitle


\chapter*{Project Description}
This project proposes to develop a statistical model for predicting
the volume of bicycle traffic per day on the Fremont bridge, by
controlling for factors such as weather, season, and day of the week.
Observed bicycle counts have been collected on the bridge since late
2012 by an automated bicycle detection system.

A report prepared for the Puget Sound Regional Council (PSRC) reviewed
various factors influencing bicycle counts, including
built-environment characteristics and time of day
factors (\cite{Bassok:2011aa}). This report corroborated findings by
\textcite{City-of-Vancouver:1999aa,Niemeier:1996aa,Parkin:2008aa} of a
relationship between bicycle counts, temperature and precipitation.
However, the report was not able to control for seasonal variation due
to the lack of a long-term, continuous dataset. A recommendation of
the report was for public agencies and researchers to consider
installing automated counters that could be used to control for
season.

In October, 2012 the City of Seattle installed automated bicycle
counters on the Fremont Bridge, a major bicycling route in Seattle.
This project will be able to take advantage of a year's worth 
of bicycle count data to draw conclusions about seasonal variations and 
other factors affecting the number of cyclists on a given day.

\chapter*{Model Variables and Data Sources}
The response variable is the daily aggregated bicycle count. An
automated detection system which uses inductive loops installed in the
sidewalk registers passing bicycles. The system then reports an hourly
aggregate count of bicycles crossing the bridge on each side, and this
count data is archived and made available on the City of
Seattle website (\cite{City-of-Seattle:aa}).

For this project, we chose to aggregate these hourly counts into daily
counts in order to focus on the question of travel and commute mode
choice in general, rather than on peak-hour volumes or similar hourly
factors. Additionally, we have divided the aggregated north-bound and
south-bound counts by two because we anticipate that most trips will
be two-way. This also addresses the issue that, while bicycle traffic
is two-way on both sides of the bridge, the inductive loops make no
distinction between bicycles traveling northbound or southbound.

Temperature and precipitation will be considered as predictors for the
bicycle count, consistent with existing literature. Historic weather
data is available through the Forecast.io public data service
(\cite{The-Dark-Sky-Company:aa}). We will also consider
seasonality, which can be conceptualized at different levels of
temporal resolution (e.g. 12 months versus 4 seasons). Other control
variables include day of the week, holidays, and whether classes at
the University of Washington are in session.

\chapter*{Additional Considerations}
We have thus far identified two potential issues that may need to be
accounted for in the regression model. The first is that the
assumption of constant variance may be violated, by virtue of working
with count data. The second issue is that there may be some dependence
between weather observations because weather for any given day is 
partially dependent on the previous day.  We will have to adjust the 
model to deal with any correlation found.

\printbibliography
\end{document}
